\documentclass{article}
\usepackage[UTF8]{ctex} % 用于中文排版
\usepackage{geometry}
\usepackage{indentfirst}
\usepackage{enumitem}

\usepackage{titling}    % 用于自定义标题页
\usepackage{graphicx}
\usepackage{float}

\usepackage{xcolor}
\usepackage{listings}

\usepackage{setspace}

% 页面几何设置
\geometry{a4paper, left=25mm, right=25mm, top=25mm, bottom=25mm}
% 取消首行缩进
\setlength{\parindent}{0pt}
% 行间距设置
\setstretch{1.5}
% 自定义字体大小
\newcommand{\fourhao}{\fontsize{14pt}{\baselineskip}\selectfont} % 四号字体
\newcommand{\xiaosihao}{\fontsize{12pt}{\baselineskip}\selectfont} % 小四号字体
\newcommand{\song}{\CJKfamily{song}}
% 设置代码块格式
\lstset{
    basicstyle = \footnotesize\ttfamily,                 % 设置行距,字体
    numbers = left,                                      % 在左侧显示行号
    numberstyle = \tiny \color{gray},                    % 设定行号格式
    keywordstyle = \bfseries \color[RGB]{40,40,255},     % 设定关键字颜色
    numberstyle = \footnotesize \color{darkgray},           
    commentstyle = \color[RGB]{0,96,96},                 % 设置代码注释的格式
    stringstyle = \color[RGB]{128,0,0},                  % 设置字符串格式
    frame = single,                                      % 不显示背景边框
    backgroundcolor = \color[RGB]{245,245,244},          % 设定背景颜色
    language=Verilog                                     % 设置语言
}
\raggedbottom   % 段落间留白, 避免排版时自动拉伸导致的行间距变化。
\begin{document}

% 封面页面
\begin{titlepage}
    \centering
    \vspace*{2cm}

    \Huge
    \textbf{课程名称:EDA 技术综合设计}

    \vspace{2cm}

    \LARGE
    设计报告名称:设计二\ 数值比较器

    \vspace{4cm}

    \centering
    \Large
    \begin{tabular}{rl}
        班级: & 通信214    \\
        姓名: & \ 王峤宇    \\
        学号: & \ 214022
    \end{tabular}

    \vfill

    \vspace{1cm}
\end{titlepage}

\newpage
% 第一部分
\section*{\fourhao 一、设计内容及原理}
\xiaosihao
\setstretch{1.5}
% 设计项目内容及设计原理,如真值表、状态表及状态转换图、文字说明等。
\subsection*{基础任务}
\textbf{设计任务}:
\subsection*{提高内容}
\textbf{设计任务}:
\subsection*{拓展任务}
\textbf{任务要求}:
% 第二部分

\section*{\fourhao 二、设计过程}
\xiaosihao
\setstretch{1.5}
% 从工程建立开始,一直到硬件调试。
% 按照基础任务、提高任务和拓展任务分别给出相应的源文件、仿真文件、约束文件
\subsection*{基础任务}
\subsection*{提高任务}
\subsection*{拓展任务}
% 第三部分
\section*{\fourhao 三、仿真结果}
\xiaosihao
\setstretch{1.5}
% 对仿真图像要有解释,要对所有的可能性进行标注及解释
% 按照基础任务、提高任务和拓展任务分别给出仿真结果
\subsection*{基础任务}
\subsection*{提高任务}
\subsection*{拓展任务}
% 第四部分
\section*{\fourhao 四、硬件验证结果}
\xiaosihao
\setstretch{1.5}
% 记录加编程器与拨码开关和发光二极管、数码管等的连接情况。记录开发板硬件验证结果,并分析其结果的正确性。
% 按照基础任务、提高任务和拓展任务分别分析
\subsection*{基础任务}
\subsection*{提高任务}
\subsection*{拓展任务}
% 第五部分
\section*{\fourhao 五、问题解决}
\xiaosihao
\setstretch{1.5}
% 设计过程中遇到的问题及解决的方法。
\subsection*{调用封装好的IP核后, 综合报错}
解决: 查看报错信息, 报错信息指向了对调用IP核接口配置处, 经过查阅相关资料, 由于自己对IP核调用的错误使用导致。
简单的IP核调用后是直接把对应的源文件导入工程下, 我误以为是直接生成对应的模块, 而为进行例化直接使用导致报错。
\subsection*{使用VS Code作为Vivado的编辑器导致Vivado的仿真报错无法进行}
解决: 对现象观察为, 通过Vivado配置编辑器更改后, Vivado默认打开VS Code编写testbench时, 直接在Vivado中进行仿真会无法进行, 会提示文件已被占用, 
针对Vivado的仿真问题有几种方法\\
1. 关闭Vivado自动打开的VS Code界面, 就可以解除占用状态, 进行仿真。\\
2. 手动建立VS Code窗口, 手动配置工作区到Vivado生成的testbench进行编辑, 此时可以在编辑testbench的同时进行Vivado界面的仿真。\\
3. 可以通过VS Code调用第三方仿真工具modelsim完成对Verilog文件的仿真。\\
% 第六部分
\section*{\fourhao 六、写出心得体会}
\xiaosihao
第一次以较为完整的流程, 包括原理梳理、报告撰写即程序编写和调试完成一次Verilog功能实现, 熟练了对Vivado的使用, 
并通过拓展任务部分的IP核设计任务, 熟悉了Vivado的IP封装工具, 并同时了解了一下Vivado提供的IP核, 如用于调试的ILA核、常用的时钟核以及FIFO核等。为之后对这些IP核的使用提供了一定的基础。
也对EGO1上的外设资源有了一定的了解, 比如和信号处理相关的, 板上FPGA芯片集成了两个12bit位宽、采样率为1MSPS的ADC, 开发板上也配备有DAC0832。\\
\end{document}
